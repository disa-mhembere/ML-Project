\documentclass[10pt]{article}

\usepackage{graphicx}
\usepackage{wrapfig}
\usepackage{url}
\usepackage{wrapfig}
\usepackage{hyperref} 
\usepackage{color}

\oddsidemargin 0mm
\evensidemargin 5mm
\topmargin -30mm
\textheight 280mm
\textwidth 160mm

\parskip 10pt 
\setlength{\parindent}{0in}

\pagestyle{myheadings} 

\title{Identifying anomalous activity in the Enron email corpus using machine learning and glocal multivariate graph statistics}

\author{Disa Mhembere (dmhembe1, dmhembe1@jhu.edu), Kunal Lillaney (klillan1, lillaney@jhu.edu)}
\date{}

\begin{document}
\maketitle

\section{Abstract}
% Clearly explain your idea.A
Graphs are quickly emerging as a leading abstraction for the representation of data flow and interactions within networks.
One high-interest area of study deals with the use of graph theory to detect attacks and anomalous activity in networks 
\cite{priebe2005scan}, \cite{park2009anomaly}, \cite{park2013anomaly}.
Additionally, the use of machine learning to detect threats within networks has become a topic of particular interest 
\cite{mahoney2003machine}, \cite{shon2005machine}, \cite{sommer2010outside}, \cite{shon2007hybrid}.

The use of glocal multivariate graph statistics \cite{mhembere2013computing} as a supplementary source of multi-dimensional 
features for anomalous network detection, has not been explored \textit{to our knowledge}. Such statistics can expose otherwise-hidden
topological attributes within networks that have the potential to improve anomalous activity detection within a network. 
We propose the use of such statistics to generate novel features for use within classification tasks for the Enron corpus 
of emails \cite{enronrepo2009}. We hope to show that the use of such statistics can help to identify periods of anomalous 
activity and associated actors.

\section{Methods} % Explain the methods you will be using and why they are appropriate.
% GT Methods
We intend to use the following statistics as supplementary attributes: Degree, Triangle Count, Clustering-coefficient,
Scan statistics, K-means clustering, Connected Components, Spectral Decomposition, Adhesion and possibly more. 
These statistics are chosen based on their ability to efficiently distill graph topological properties into discrete
values that are usable by ML techniques. We will use the igraph \cite{igraph2006} package to create the time-series
graphs and compute statistics. We will parse and transform the time series graphs into an ML ingestible format. \newline

%ML Methods
Since we can correlate events in the news with time, we hope to generate classification labels based on events
surrounding reported Enron fraudulent activities. Literature \cite{shon2007hybrid}, \cite{shon2005machine} has 
led us to believe that SVMs may be suitable from such a classification task. We propose the use of a weighted kNN to 
identify malicious actors in the network at different time points since we conjecture the features of malicious users will 
have high relative `similarity'.

\section{Resources}
% What resources will you use and how will you get them?
We will use the following resources to generate our dataset: (i) \texttt{http://www.cs.cmu.edu/~./enron/} -- Contains 0.5M 
emails from senior management of Enron, (ii) \texttt{http://cis.jhu.edu/~parky/Enron/enron.html} -- Scan Statistics on
Enron graphs. We will use custom scripts to create ML-ingestible instances/features.

\section{Milestones}
\subsection{Must achieve}
Able to correlate specific reported events in the news with anomalous email activity. 
\subsection{Expected to achieve}
Identify the different activities as anomalous or benign events.
\subsection{Would like to achieve}
Determine specific actors as anomalous at varying time points through hybrid weighted kNN and SVM techniques.

\section{Final Writeup}
% What will appear in the final writeup. 
% Chose not to use bullets because of space
Our final writeup will include (i) A survey of graph statical algorithms pertinent to classification tasks, 
(ii) A survey of machine learning techniques used to detect anomalies in networks, 
(iii) A cost-benefit analysis between methods in \textit{(ii)} and our approach,
(iv) Detailed description of our methods and associated novelty,
(v) The classification results of anomalous activity on the Enron dataset,
(vi) The classification of individual actors during times of anomalies,
(vii) Future work and Conclusion.
 
\section{Bibliography}
\bibliographystyle{IEEEtran}
{\small
\bibliography{template.bib}
}

\end{document}
