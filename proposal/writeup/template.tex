\documentclass[10pt]{article}

\usepackage{graphicx}
\usepackage{wrapfig}
\usepackage{url}
\usepackage{wrapfig}
\usepackage{hyperref} 
\usepackage{color}

\oddsidemargin 0mm
\evensidemargin 5mm
\topmargin -20mm
\textheight 240mm
\textwidth 160mm

\parskip 10pt 
\setlength{\parindent}{0in}

\pagestyle{myheadings} 

\title{Identifying anomalous activity in the Enron email corpus using machine learning and glocal multivariate graph statistics}

\author{Disa Mhembere (dmhembe1, dmhembe1@jhu.edu), Kunal Lillaney (klillan1, lillaney@jhu.edu)}
\date{}

\begin{document}
\maketitle

\section{Abstract}
% Clearly explain your idea.A
Graphs are quickly emerging as a leading abstraction for the representation of data flow and interactions within networks.
One high-interest area of study deals with the use of graph theory to detect attacks and anomalous activity in networks 
\cite{priebe2005scan}, \cite{park2008scan}, \cite{park2009anomaly}, \cite{park2013anomaly}.
Additionally, the use of machine learning to detect threats within networks has become a topic of popular topic of interest \cite{mahoney2003machine}
%(http://www.sciencedirect.com/science/article/pii/S0020025507001648), (http://ieeexplore.ieee.org/xpls/abs_all.jsp?arnumber=5504793&tag=1), (http://ieeexplore.ieee.org/xpls/abs_all.jsp?arnumber=1495950), (https://msdn.cs.fit.edu/media/TechnicalReports/cs-2003-13.pdf)
.

The use of glocal multivariate graph statistics \cite{mhembere2013computing} as a supplementary source multi-dimensional features for anomalous network detection,
has not been explored to our knowledge. Such statistics can expose otherwise-hidden topological attributes within networks that have the potential 
to improve anomalous activity detection within a network. We propose the use of such statistics to generate novel features for use within 
classification tasks for the Enron corpus of emails (cite). We hope to show that the use of such statistics can help to identify periods of anomalous 
activity and associated actors.

\section{Methods} % Explain the methods you will be using and why they are appropriate.
% GT Methods
We intend to use the following statistics as supplementary attributes: Degree, Triangle Count, Clustering-coefficient,
Scan statistics, K-means clustering, Connected Components, Spectral Decomposition, Adhesion and possibly more. These statistics are
chosen based on their ability to efficiently distill graph topological properties into discrete values that are usable by ML techniques.
We will use the igraph (cite) package to create the time-series graphs and compute statistics. 
We will parse and transform the time series graphs into an ML ingestible format. \newline

%ML Methods
Since we can correlate events in the news with time We intend to use 


\section{Resources}
% What resources will you use and how will you get them?
We will use the mentioned resources to generate our dataset.
\begin{enumerate}
  \item http://www.cs.cmu.edu/~./enron/ - Contains 0.5M emails from senior management of Enron. 
  \item http://cis.jhu.edu/~parky/Enron/enron.html - Scan Statistics on Enron Graphs.
\end{enumerate}
\section{Milestones}
\subsection{Must achieve}
Able to correlate time periods and actors into different activities. 
\subsection{Expected to achieve}
Identify the different activities as anomalous events or everyday ones. 
\subsection{Would like to achieve}
Decide whether an anamoly is malicious or benign.

\section{Final Writeup}
% What will appear in the final writeup.

\section{Bibliography}
\bibliographystyle{IEEEtran}
{\small
\bibliography{template.bib}
}

\end{document}
