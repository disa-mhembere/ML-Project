
\documentclass[11pt,letterpaper]{article}
\usepackage{naaclhlt2010}
\usepackage{paralist}
\usepackage{times}
\usepackage{latexsym}
\setlength\titlebox{6.5cm}    % Expanding the titlebox

\title{Identifying Anomalous Activity the Enron Email \\
				Corpus using Glocal Multivariate Graph Statistics}

\author{Disa Mhembere\\
  Department of Computer Science\\
  Johns Hopkins University\\
  {\tt dmhembe1@jhu.edu}
  \And
  Kunal Lillaney \\
  Department of Computer Science\\
  Johns Hopkins University\\
  {\tt lillaney@jhu.edu}}

\date{}

\begin{document}
\maketitle
\begin{abstract}
Graphs are quickly emerging as a leading abstraction for the representation of data
flow and interactions within networks. One high-interest area of study deals with 
the use of graph theory to detect attacks and anomalous activity in networks 
\cite{priebe2005scan,park2009anomaly,park2013anomaly}.
Additionally, the use of machine learning to detect threats within networks 
has become a topic of particular interest 
\cite{mahoney2003machine,shon2005machine,sommer2010outside,shon2007hybrid} within
cyber security and machine learning circles alike.

The use of glocal multivariate graph statistics \cite{mhembere2013computing} as 
a supplementary source of multi-dimensional features for anomalous network detection,
has not been explored \textit{to our knowledge}. Such statistics can expose 
otherwise-latent topological attributes within networks that have the potential 
to improve anomalous activity detection within a network. 

We use of such statistics to generate novel features for use within
classification tasks for the Enron corpus of emails \cite{enronrepo2009}.
We hope to show that the use of such statistics can help to identify periods of anomalous 
activity and associated actors.

\end{abstract}


\section{Introduction}
Graphs are an intuitive high-level abstraction of interaction in any setting where
there is information flow are several participants in the exchange of it.

\section{Background}
Todo

\section{Design}
This section describes our choice of glocal graph statistics and discusses their suitability
to the task of detection anomalous actors in the Enron email corpus.
We further discus our choice of Machine Learning algorithms. Lastly, we discuss our metrics
of success and evaluation for our implementations.

\subsection{Graph Statistics}
Todo

\subsection{Machine Learning Machinery}
Literature \cite{shon2007hybrid,shon2005machine} has 
led us to believe that SVMs may be suitable from such a classification task. 
We propose the use of a weighted kNN to identify malicious actors in the network 
at different time points since we conjecture the features of malicious users will 
have high relative `similarity'.


\section{Methods}
In this section we describe both data acquisition phase and the learning algorithm
implementation and the challenges we faced in the pursuit of both.

\subsection{Data Acquisition}
To realize our goals we obtain raw data from two sources:
\begin{inparaenum}[\itshape(i)]
\item http://www.cs.cmu.edu/$\sim$./enron/. This resource contains over 0.5 million
emails from senior management within Enron and 
\item http://cis.jhu.edu/parky/Enron/enron.html. This is a resource from where we obtain
time-series graphs of Enron email activity for a 189 week period between 1998 and 2002.
Here each graph represents a week of email interaction between $\sim$150 to executives
at Enron.
\end{inparaenum}

\subsubsection{Raw email Feature Extraction}

\subsubsection{Challenges: Raw email Feature Extraction}
We faced several

- Several email addresses to a single user
- Non-uniform email addresses generated in raw text
- Auto-generated emails


\subsubsection{Time Series Feature Extraction}
Todo
\subsubsection{Challenges: Feature Extraction}
Todo
\subsubsection{Regularization}
Todo

\section{Experiment Design}
Todo
\section{Results}
Todo

\section{Comparison to proposal}
We did achieve all we set out to accomplish!

\section{Conclusion}
Todo



\section*{Bibliography}
\bibliographystyle{IEEEtran}
\bibliography{writeup.bib}
%\bibliographystyle{acl}
%\bibliography{references.bib}
\end{document}
